% Chapter: Movies page

\section{Summary}
The Movies page has the most functionality and most complex functionality. Also, 
to avoid any confusion, there's nothing film-related here. A ``movie'' is the 
sequencing which has taken place on a single stage within a sequencing run. Up 
to four independent stages make up a ``run''. Now that that's out of the way….

\section{Table}
Each row in the movie table will have a column for each of the following properties 
belonging to the movie object. Rows of movie data should be sorted by date. The 
cells of some columns serve as buttons to open a pop-up window.

\begin{table}[h]
    \begin{tabularx}{\textwidth}{l|l|X}
\toprule
 Column Name   & Property Name   & Pop-up Window                           \\
\midrule
 Movie Id      & \texttt{movie\_id}        & n/a                                     \\
 Run Id        & \texttt{run\_id}          & Movies On Run \texttt{run\_id}                    \\
 Date          & \texttt{date\_str}        & n/a                                     \\
 Submitted By  & \texttt{user\_name}       & n/a                                     \\
 Instrument    & \texttt{instrument\_name} & n/a                                     \\
 Sample Type   & \texttt{sample\_type}     & n/a                                     \\
 Sample Name   & \texttt{sample\_name}     & n/a                                     \\
 Shares        & \texttt{num\_shares}      & Users With Access To Movie \texttt{movie\_id}     \\
 Analyses      & \texttt{num\_analyses}    & Analyses Associated With Movie \texttt{movie\_id} \\
 Backup Device & \texttt{backup\_device}   & n/a                                     \\
\bottomrule
\end{tabularx}
    \caption{Movies table}
\end{table}

\section{Pop-up windows}

\subsection{Movies on run run\_id}
List of movies that ran on run \texttt{run\_id}. Rows should be the same as in the main 
Movies table. This functionality could also be implemented by a filter being applied 
to the main Movies table.

\subsection{Sharing of movie movie\_id}
List of users who have been granted access to this movie. Rows should contain the 
same fields as those in the main Users table. First, get an 
\hyperref[section:updatingcollectionaccessrules]{up-to-date list of access rules} 
to determine which users have access to this movie. Then, use this information 
to filter out the appropriate \texttt{user} records.

\subsection{View movie directory}
\subsection{Analyses associated with movie movie\_id}
\section{Actions}
\subsection{Share}
\subsection{Download}
\subsection{Upload}
\subsection{Unshare}
\subsection{Remove}

