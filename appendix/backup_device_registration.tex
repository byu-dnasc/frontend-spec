% Appendix: Backup Device Registration

\section{Backup Device Registration}

The register backup device pop-up window contains multiple dynamic elements.

\subsection{Globus collection search}

A search box is provided for the user to search for Globus collections. The search
results are displayed in a dropdown menu. Each search result should include the
collection name and the collection owner. The JSON response from the Globus
API should be parsed to extract the collection name, owner, and endpoint
id. \linkglobusendpointsearch{Link to API documentation for searching collections}.

\subsection{Selecting a path using the Globus ``Browse Endpoint'' helper page}

Globus provides a ``helper page'' for selecting a path on an Globus collection.

There is a dedicated pop-up window for registering backup devices. It functions
to collection information from the user about a storage device and its location,
validate the information, place a ``cookie'' on the backup device to identify it 
for future transfers, and add a record of the device to the database.

\subsubsection{Step 1: identify backup device}

Once the path of the device on the endpoint is determined, a ``cookie'' 
should be created and placed on the backup device via Globus Transfer. This cookie 
should consist of an empty file with a unique name. In the event of a download, this 
cookie will be used to identify the backup device.

Upon successful registration, the user-provided name for the backup device and the 
filename of the cookie should be stored in the database. The device names will be 
used to populate a dropdown menu in the ``Create Backup'' pop-up window for the
user to select the backup device to use for the transfer.

Backup device data should be stored in a database on the website backend, not in
the Globus Search index.

First, a Globus endpoint must be identified. The user should be given the option
to either select a favorite endpoint or search for an endpoint. If the user
selects a favorite endpoint, the endpoint id will be displayed in a text box
and the path selection step will be enabled. If the desired endpoint is not
listed as a favorite, the user can search for the endpoint.
The user should be prompted to provide
a string for querying the Globus endpoint search API. The names of the resulting 
list of endpoints will be displayed in a dropdown listing. The user will select
the desired endpoint from this list.

