% Chapter: Globus APIs and concepts

The new DNASC web pages will get and put data using REST APIs administered by Globus, 
a platform with services geared toward managing research data. Globus APIs return JSON 
data representing various things such as tabular data, file system contents, and file 
operations.
\section{Globus Collections}
The DNASC maintains a Globus ``Collection'' which is essentially a file system which 
can be made accessible to any Globus user using the Globus platform (in some documentation, 
you may see the term ``Endpoint'' used synonymously with ``Collection'').

\section{File operations}
A simple, straightforward objective which will be useful in implementing the new web 
pages is the \linkglobusfilelisting{listing of files} in the DNASC Globus collection.

\section{Task submission}
The Movies page will need to initiate file transfers via the Globus Transfer API. This 
can be accomplished by submitting requests for files on the DNASC Globus collection 
to be transferred to a Globus collection on a DNASC PC (or the opposite direction).

It may be useful if the DNASC website could \linkglobusbrowseendpoint{redirect} the user 
to the Globus web app.

\section{Task management\label{appendix:globusapis:taskmanagement}}
From the \linkglobustaskmonitoring{documentation}, ``Transfer and delete are asynchronous 
operations, and result in a background task being created. The task id is returned from 
successful submission, and can be used to monitor the progress of the task''. Task 
monitoring is needed in the Active Transfers page.

\section{Search}

\subsection{The search index}
The new website needs to keep track of file metadata (such as paths, formats and sizes) 
as well as tabular data about lab operations and users. Luckily, Globus provides a database 
service accessible through its Search API. An “index” has been prepared to support the new 
DNASC web functionality. The index is maintained by a backend running on a server administered 
by BYU ORC.

\subsection{Querying}
Most of the data that will be displayed in tables to the user will come from the search 
index. Some fields can simply be read from the database, but others must be computed by 
executing a query. In contrast to other types of databases, a search index is optimized 
for searching and retrieving data based on keywords and phrases rather than structured 
queries.

\section{ACL management}
Globus users can be granted access to specific files and directories on a given collection. 
This is accomplished by maintaining a set of access rules. When a user of the new website 
goes to share a new dataset with a DNASC customer, a new access rule will be created granting 
the specific customer access to their data by specifying a directory or file path.
