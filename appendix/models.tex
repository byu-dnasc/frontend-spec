% tables are "floating" out of place. this is a default behavior for report and srcreprt

The following tables containing name-value pairs represent objects which will 
be used on the frontend. Some objects are generated using the search index and 
others come from input collected by a user of the website.

\section{Frontend objects}
These objects are put together on the frontend.

\subsection{\texttt{download\_request}}
A \texttt{download\_request} is an instance of a Globus \linkglobustransferitem
{\texttt{transfer\_item}}. It follows the \linkglobustransferrecursive{rules} 
laid out for transferring a directory.

\begin{table}[h] % Placement options: h (here), t (top), b (bottom), p (page)
    \input{table/download_request_model.tex}
    \caption{Download Request Model. See the documentation for filter rules \linkglobusfilterrules{here}.}
\end{table}

\subsection{\texttt{upload\_request}}
An \texttt{upload\_request} is an instance of a Globus \linkglobustransferitem
{\texttt{transfer\_item}}. It follows the \linkglobustransferrecursive{rules} laid 
out for transferring a directory.

\begin{table}[h] % Placement options: h (here), t (top), b (bottom), p (page)
    \begin{tabularx}{\textwidth}{l|X}
\hline
 Property Name           & Description                                                          \\
\hline
 \texttt{recursive}               & Always true                                                          \\
 \texttt{destination\_endpoint\_id} & DNASC\_EP\_ID                                                          \\
 \texttt{source\_endpoint\_id}      & User specified endpoint ID                                           \\
 \texttt{source\_path}             & A path prefixed with the path of a drive to a movie directory        \\
 \texttt{destination\_path}        & ``/''                                                                \\
 \texttt{verify\_checksum}         & false for now                                                        \\
 \texttt{label}                   & A string containing the total number of bytes across all movie files \\
 \texttt{filter\_rules}            & A list of filter rules                                               \\
\hline
\end{tabularx}
    \caption{Upload Request Model. See the documentation for filter rules \linkglobusfilterrules{here}.}
\end{table}

\subsection{\texttt{transfer\_task}}
A \texttt{transfer\_task} object is based on values found in a \linkglobustaskdocument
{Task document} obtained using \hyperref{appendix:globusapis:taskmanagement}{this} Globus API 
described in Appendix \ref{appendix:globusapis}.

\begin{table}[h] % Placement options: h (here), t (top), b (bottom), p (page)
    \begin{tabularx}{\textwidth}{l|X}
\hline
 Property Name       & Description                                                                                                              \\
\hline
 \texttt{transfer\_type}       & ``Download'' if the DNASC Globus endpoint is the source, else ``Upload''. Use task document property source\_endpoint\_id. \\
 \texttt{bytes\_transferred}   & Current total number of bytes transferred. Use the value of task document property \texttt{bytes\_transferred}.                    \\
 \texttt{transfer\_size}       & Read from label property in the task document                                                                            \\
 \texttt{progress\_percentage} & bytes\_transfered / \texttt{transfer\_size} * 100, rounded to nearest integer                                                       \\
\hline
\end{tabularx}
    \caption{Transfer Task Model.}
\end{table}

\section{Database objects}

\subsection{\texttt{movie}}
\begin{table}[h] % Placement options: h (here), t (top), b (bottom), p (page)
    \begin{tabularx}{\textwidth}{l|X}
\hline
 Property Name   & Description                                                     \\
\hline
 \texttt{movie\_id}        & The movie ID                                                    \\
 \texttt{run\_id}          & The run ID                                                      \\
 \texttt{date\_str}        & The time at which the movie was created                         \\
 \texttt{user\_name}       & The name of user who submitted the movie                        \\
 \texttt{instrument\_name} & ``Sequel II'' or ``Revio''                                      \\
 \texttt{sample\_type}     & ``HiFi'' (default), ``Iso-Seq'', ``MAS-Seq'', or ``Amplicons''  \\
 \texttt{sample\_name}     & The name of the sample                                          \\
 \texttt{num\_shares}      & The number of access rules granting access to this movie's path \\
 \texttt{num\_analyses}    & The number of analyses associated with this movie               \\
\hline
\end{tabularx}
    \caption{Movie Model.}
\end{table}

\subsection{\texttt{analysis}}

\section{Real-time State of Globus Collection}

As often as possible, the state of the Globus collection should be obtained using a Globus API 
(by listing files or access rules, etc.) as opposed to storing information about the state in 
the database. This is because the state of the Globus collection can be changed at any time, 
DNASC website users not being the only agents of this change.

This being the case, the following functions must be implemented:

\subsection{is\_online}

This function would make sense as a method in a movie class. It returns True if a given movie 
is available in the Globus collection. Arguments to this function would be the collection ID 
and the path of the movie. Add specific API call.

\subsection{get\_analyses}

This function would make sense as a method in a movie class. It returns a list of analysis 
objects (if any). The implementation is relatively complex as it involves accessing file 
listings in the Globus collection. Call 