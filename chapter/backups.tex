%  Chapter: Backups page

\section{Overview}

\noindent This chapter will describe the contents of the backup pop-up window.
The backup pop-up window will facilitate the backing up of sequencing data files 
generated in the DNASC. It should allow the user initiate the copying of
files associated with the movie to a backup device selected by the user.
It should also allow the user to register a new backup device. The pop-up window 
is opened by clicking the cell in the ``backup'' column of the movies table for a 
given movie.

\subsection{Backup workflow}
Movie files are initially stored on systems administered by the BYU Office of 
Research Computing (ORC). An essential task for DNASC technicians is to copy sequencing 
data files from ORC storage to local hard drives. Once backed up, the files can be 
removed from ORC storage to free up space. Note that the workflow as described below
allows for new backup to replace a previous backup.

\section{Content}

The contents of the backup pop-up window can be partitioned into three sections: 
``Movie Details'', ``Backup Details'', and ``Create Backup''. The pop-up window should 
display all three sections in the order listed above, including the section name 
(formatted as a title) and contents. The contents of each section are described below. 

\subsection{Movie details section}

The movie details section consists of the following items:
\begin{itemize}\itemsep1pt
  \item Movie id
  \item Sample name
  \item File size
\end{itemize}

\subsection{Backup details section}

The contents of the backup details section should vary depending on 
whether or not the movie has been backed up before. If the movie has not been 
backed up before, the backup details section should consist of the following
string: 

\vspace{3mm}
``No backups have been created for this movie.''
\vspace{3mm}

\noindent If the movie has been backed up before, the Backup Details section
should consist of the following name-value pairs belonging to the \texttt{movie} object:

\begin{itemize}\itemsep1pt
  \item ``Backup Device: \texttt{backup\_device}''
  \item ``Backup Date: \texttt{backup\_date}'' 
\end{itemize}

\subsection{Create backup section}

The create backup section allows the user to create a backup by setting up the
copying of the given movie to a given backup device. The section should consist 
of the following items in the order in which they are listed:

\begin{itemize}\itemsep1pt
  \item A subsection for selecting a Globus collection on which to discover backup
  devices, labeled ``Select Globus Collection'' (see below).
  \item A dropdown menu of backup devices available on the selected Globus 
  collection, labeled ``Backup Device''.
  \item A button labeled ``Initiate Transfer''.
  \item A button labeled ``Register Backup Device''.
\end{itemize}

\subsubsection{Select Globus collection subsection}

The select Globus collection subsection provides two options to the user for selecting
a Globus collection on which to discover backup devices. The subsection should consist
of the following items:

\begin{itemize}\itemsep1pt
    \item A dropdown menu of the user's favorite collections, labeled ``Favorite
    Collections''
    \item \emph{Consider using the browse endpoint helper page instead of searching collections.} 
    A search box for searching all Globus collections, labeled ``Search
    Collections''. Each search result should include the collection name and the 
    collection owner. 
\end{itemize}

\subsubsection{Element availability}

Due to dependencies between the elements in the create backup section, some elements
should not (or can not) be made available until certain conditions are met. All elements 
should be visible at all times, but some elements should be disabled initially. The 
conditions under which each respective element should be enabled are as follows:

\begin{itemize}\itemsep1pt
  \item The favorite collection dropdown menu and search collections box are enabled at all times
  \item The backup device dropdown menu is available only when an collection has been
  selected
  \item The initiate transfer button is available only when a backup device has been
  selected
  \item The register backup device button is available at all times 
\end{itemize}


