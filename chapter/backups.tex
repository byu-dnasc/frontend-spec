%  Chapter: Backups page

\section{Overview}

\vspace{3mm}
\emph{``The cloud is just someone else's computer.'' -Anonymous}
\vspace{3mm}

\noindent This chapter will describe the contents of the backup pop-up window.
The backup pop-up window will facilitate the backing up of sequencing data files 
generated in the DNASC. It should allow the user initiate the copying of
files associated with the given movie to a backup device selected by the user.
It should also allow the user to register a new backup device. The pop-up window 
is opened by clicking the cell in the ``Backup'' column of the movies table for a 
given movie.

\subsection{Backup workflow}
Movie files are initially stored on systems administered by the BYU Office of 
Research Computing (ORC). An essential task for DNASC technicians is to copy primary 
movie files from ORC storage to local hard drives. Once backed up, the files can be 
removed from ORC storage to free up space. Note that the workflow as described below
allows for new backup to replace a previous backup.

\section{Contents}

The contents of the backups pop-up window can be partitioned into three subsections: 
``Movie Details'', ``Backup Details'', and ``Create Backup''. The pop-up window should 
display all three subsections in a single listing in the order listed above, including 
the subsection name (formatted as a title) and contents. The contents of each subsection 
are described below. 

\subsection{Movie Details}

The movie details subsection consists of the following items:
\begin{itemize}\itemsep1pt
  \item Movie id
  \item Sample name
  \item File size
\end{itemize}

\subsection{Backup Details}

The contents of the backup details subsection should should vary depending on 
whether or not the movie has been backed up before. If the movie has not been 
backed up before, the backup details subsection should consist of the following
string: 

\vspace{3mm}
``No backups have been created for this movie.''
\vspace{3mm}

\noindent If the movie has been backed up before, the Backup Details subsection
should consist of the following name-value pairs:

\begin{itemize}\itemsep1pt
  \item ``Backup Device'' and the value of movie property \texttt{backup\_device}
  \item ``Backup Date'' and the value of movie property \texttt{backup\_date}
\end{itemize}

\subsection{Create Backup}

The create backup subsection allows the user to create a backup by setting up the
copying of the given movie to a given backup device. The subsection should consist 
of the following items:

\begin{itemize}\itemsep1pt
  \item A dropdown selection menu of registered GCP endpoints, labeled ``Endpoint''
  \item A dropdown selection menu of backup devices available on the selected 
  endpoint, labeled ``Backup Device''
  \item A button labeled ``Initiate Transfer'' (see appendix 
  \ref{appendix:createbackup})
  \item A button labeled ``Register Backup Device'' (see chapter 
  \ref{chapter:registerbackupdevice})
\end{itemize}

\subsubsection{Element availability}

Due to dependencies between the elements in the create backup subsection, some elements
should not (or can not) be made available until certain conditions are met. All elements 
should be visible at all times, but some elements should be disabled initially. The 
conditions under which each respective element should be enabled are as follows:

\begin{itemize}\itemsep1pt
  \item The endpoint selection menu is enabled at all times
  \item The backup device selection menu is available only when an endpoint has been
  selected
  \item The initiate transfer button is available only when a backup device has been
  selected
  \item The register backup device button is available at all times 
\end{itemize}


