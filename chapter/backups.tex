% Chapter: Backups page

\section{Overview}

\vspace{3mm}
\emph{``The cloud is just someone else's computer.'' -Anonymous}
\vspace{3mm}

A new page will be created to facilitate the backing up of sequencing data 
files generated in the DNASC. These files are initially stored on systems 
administered by the Office of Research Computing (ORC). An essential task 
for DNASC technicians is to copy primary data files from ORC storage to local 
hard drives. Once backed up, the files can be removed from ORC storage to 
free up space.

\section{Components}

The backups page will consist of a table of movies and a button to open a
pop-up window to collect information to register a new backup device. 

\subsection{Table: ``Awaiting Backup''}

A table on the backups page will display movies that have not yet been backed
up, sorted by the date the movie was created. Only the following fields should 
be displayed:

\begin{itemize}\itemsep1pt
    \item Movie name
    \item Movie creation date
    \item Movie size
    \item Backup device
    \item Backup status
\end{itemize}

\noindent Included in each row will be a button ``Create Backup'' to initiate 
the backup process for the respective movie.

\section{Actions}

\subsection{Register Backup Device}

The register backup device action is used to register a new backup device.

\subsubsection{Movitation}
The registration of backup devices simplifies the transfer process in that the 
backup device object serves as an abstraction of the Globus endpoint and path 
which are the transfer destination (or, in the seldom case in which data on 
a backup device needs to be uploaded, the source).

\subsubsection{Implementation}
Following details must be collected to register a backup device:
\begin{itemize}\itemsep1pt
    \item Name
    \item Globus endpoint id
    \item Globus endpoint path. \emph{Given the endpoint id, there should be some pre-built logic to send the user to
the app.globus.org to browse the endpoint and select a path (https://docs.globus.org/modern-research-data-portal/mrdp-description/).}
\end{itemize}

\noindent 
Once the path of the device on the endpoint is determined, a ``cookie'' should be 
created and placed on the backup device via Globus Transfer. This cookie should
consist of an empty file with a unique name. In the event of a download, this 
cookie will be used to identify the backup device.

Upon successful registration, the user-provided name for the backup device and the 
filename of the cookie should be stored in the database. The device names will be 
used to populate a dropdown menu in the ``Create Backup'' pop-up window for the
user to select the backup device to use for the transfer.

Backup device data should be stored in a database on the website backend, not in
the Globus Search index.

\subsection{Create Backup}

The create backup action is used to copy a given movie to a backup device on a Globus
endpoint. The action is initiated by clicking the ``Create Backup'' button available 
in each row of the ``Awaiting Backup'' table. The button opens a pop-up window to
allow the user to select an endpoint and a backup device available on the endpoint 
to use for the transfer.

\subsubsection{Implementation}

% too detailed. put this paragraph somewhere else
When the ``Create Backup'' action is initiated, two lists should be created to populate
the dropdown menus in the pop-up window. The first list should contain all endpoints
which have been used before. If the user has not used any endpoints before, the list
should be empty. The second list should contain all backup devices available on the
selected endpoint. The frontend will use the Globus Transfer API to 
\href{appendix:identifybackupdevices}{identify all backup devices} available on the 
user-selected endpoint.

The user can select an endpoint and a backup device from the respective dropdown menus.
Once this is accomplished, the transfer can be submitted.
