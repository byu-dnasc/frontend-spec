\section{Overview}

This chapter will describe modifications and new features that should be added to the
``Users'' page of the current DNASC website. The current system that the the users 
page interacts with should be expanded to include a new type of user (user with role 
``External'') and a new field for Globus usernames. This will require modifications 
to the ``New User'' and ``Edit User'' forms and the user table on the users page. Another 
minor change will be to enable technicians to manage external user information.
These additions to the users system will support the sharing functionalities introduced 
by the new features described in this specification. 

\subsection{Discretionary implementation}

Several decisions regarding the implementation of the new features described in this 
chapter should be made by the web development team due to the superior information and 
expertise at their disposal. This chapter will first describe the essential features 
that must be implemented. Then, it will discuss design choices that will be left to the 
discretion of the web development team.

\section{Essential features}

\subsection{Data model changes}

The user data model should be modified to include:
\begin{itemize}\itemsep1pt
    \item A new field ``Globus Username''. This field should be required for 
external users, but optional to all others.
    \item A new role ``External'' should be added to the existing set of values
    available for the role field.
\end{itemize}

\subsection{Actions available to technicians}

The ``New User'' button should become available to all technicians. However, it should
open a form which can only be used to collect the information required to create an
external user. The edit and delete user buttons should also become available to 
technicians, but only for external user records.

\section{Design choices}

\subsubsection{One user form, or two?}

Whether to create two discrete forms for creating external users and other
users, or to create a single form for creating all users.

\subsubsection{One button, or two?}

This question may rely on the answer to the previous question.
Whether to create a separate button for creating external users, or to
use a single ``New User'' button for all users.

\section{External user forms}

The following fields are the only ones that should be collected for external users:

\begin{itemize}\itemsep1pt
    \item Firstname
    \item Lastname
    \item Email
    \item Globus username
\end{itemize}

\section{External user registration pop-up window}

A new pop-up window should be created consisting of a form to collect the information 
to create a new external user. This pop-up window could be similar to the existing user 
registration form, but with the following differences:

Firstname, lastname, email, Globus username
\begin{itemize}\itemsep1pt
    \item A new field ``Globus Username'' should be added to the form.
    \item A new field ``One-off'' should be added to the form. This field should be
    a checkbox or toggle switch. Records of one-off users should be automatically 
    deleted after one year. A description should be added to the form to explain this:
    ``One-off users will be automatically deleted after one year.''
    \item The ``Net ID'' field should not be collected. The field could be populated 
    with some default value when displayed (like ``n/a'').
    \item The field ``Department'' should also not be collected. This field could be
    populated with some default value when displayed (like ``n/a'').
    \item The value of the ``Role'' field should be automatically populated with 
    ``External''.
\end{itemize}

\subsection{Edit user}

Technicians should be able to edit the information of external users.

The edit user action is already implemented. However, it is currently only available
to the user who is to be edited. This action needs to be made available to all 
technicians for all external users. Additionally, the edit user pop-up should include 
a new button to allow technicians to delete external users.

\section{Future improvement}
The current user registration system is integrated with BYU's central 
authentication system (CAS). This means that external users are currently not 
able to create accounts. However, this could be a useful feature in the future 
if it became desirable for external users to create their own account with the DNASC.
For now, though, it seems that requiring external users to register themselves with 
the DNASC via the website would too tedious for the mere purpose of collecting their
Globus username.

