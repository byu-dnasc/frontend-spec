\section{Overview}

This chapter will describe modifications and new features that should be added to the
``Users'' page of the current DNASC website. The current system that the the users 
page interacts with should be expanded to include a new type of user (user with role 
``External'') and a new field for Globus usernames. External user records would not 
represent actual users of the website, rather they are simply information to be stored 
for future reference. The addition of these records of external users will support the 
sharing functionalities introduced by the new features described in this specification.
These new features will require modifications to the ``New User'' and ``Edit User'' 
forms and the user table on the users page. Another minor change will be to enable 
technicians to manage external user information.

\subsection{Discretionary implementation}

Several decisions regarding the implementation of the new features described in this 
chapter should be made by the web development team due to the superior information and 
expertise at their disposal. This chapter will first describe the essential features 
that must be implemented. Then, it will discuss design choices that will be left to the 
discretion of the web development team.

\section{Essential features}

\subsection{Data model changes}

The user data model should be modified to include:
\begin{itemize}\itemsep1pt
    \item A new field ``Globus Username''. This field should be required for 
external users, but optional to all others.
    \item A new role ``External'' should be added to the existing set of values
    available for the role field.
\end{itemize}

\subsection{Actions available to technicians}

The ``New User'' button should become available to all technicians. However, it should
open a form which can only be used to collect the information required to create an
external user. The edit and delete user buttons should also become available to 
technicians, but only for external user records.

\section{External user forms}

The following fields are the only ones that should be collected for external users:
\begin{itemize}\itemsep1pt
    \item Firstname
    \item Lastname
    \item Email
    \item Globus username
\end{itemize}
The rest of the fields in the user model should display a default value like ``n/a'',
except for the role field, which should be set to ``External''.

\section{Design choices}

\subsection{One user form, or two?}

The current new user form (currently only available to system administrators) can be 
used to create a user with any role. However, technicians should not be given access 
to this form, at least not without modifications. An alternative to this would be to 
create a second form which only collected the information required to create an external 
user. This form would be available to technicians and system administrators.
It is left to the discretion of the web development team to decide whether to create
a second form which can only be used to create external users, or to modify the existing
form such that, when accessed by a technician, it only allows the creation of external
users. In the case of the latter option, the current new user form would need to be 
modified to include a field for Globus username, as well as to include ``External'' 
as a possible value for the role field. In both cases, all users should have a field
for Globus username.

\subsection{One button, or two?}

This question is only relevant if the answer to the first question is two forms.
If there are two forms (one for creating only external users, and one for creating 
any type of user), then there would be the opportunity for two buttons:
\begin{itemize}\itemsep1pt
    \item A ``New User'' button which opens the form for creating any type of user.
    \item A ``New External User'' button which opens the form for creating only 
    external users.
\end{itemize}
Or, a single button could be used which would open a different form depending on the
role of the user who clicked the button. Either way, only one button would be available 
to technicians. Having only one button would allow for a simpler user interface. But if 
having two buttons would be significantly easier to implement, then that would be 
acceptable.