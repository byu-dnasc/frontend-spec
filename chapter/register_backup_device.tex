% Chapter: Register Backup Device Pop-up

\section{Overview}

This chapter will describe the contents of the register backup device pop-up 
window. This pop-up will facilitate the registration of a new backup device
for use in the DNASC. It is opened by clicking the ``Register Backup Device''
button in the ``Create Backup'' subsection of the backups pop-up window.
Once registered, the backup device will be available to be selected by the
user as the destination for a backup transfer.

\subsection{Movitation}
The registration of backup devices simplifies the transfer process in that the 
backup device object serves as an abstraction of the Globus endpoint and path 
which are the transfer destination (or, in the seldom case in which data on 
a backup device needs to be uploaded, the source).

\section{Register backup device form}

The register backup device pop-up window consists of an interactive form with
which the user can select, search and input information about a new backup 
device. The form should contain three sections, each dedicated to collecting 
one of the following three pieces of information needed to identify and register 
a backup device.

\begin{itemize}\itemsep1pt
    \item Globus Collection
    \item Path on the Globus Collection. \emph{Given the endpoint id, there should be some 
    pre-built logic to send the user to app.globus.org to browse the endpoint and 
    select a path 
    (https://docs.globus.org/modern-research-data-portal/mrdp-description/).}
    \item Backup device name (arbitrary string provided by user)
\end{itemize}

\subsection{Registration Process}

\subsubsection{Step 1: Endpoint selection}

First, a Globus endpoint must be identified. The user should be given the option
to either select a favorite endpoint or search for an endpoint. If the user
selects a favorite endpoint, the endpoint id will be displayed in a text box
and the path selection step will be enabled. If the desired endpoint is not
listed as a favorite, the user can search for the endpoint.
The user should be prompted to provide
a string for querying the Globus endpoint search API. The names of the resulting 
list of endpoints will be displayed in a dropdown listing. The user will select
the desired endpoint from this list. At this stage, the user should be given the
opportunity to save this endpoint as a favorite for future use.

\subsubsection{Step 2: Path selection}

Once an endpoint has been selected, the user will be prompted to select a path
on the endpoint. The user will be redirected to app.globus.org to browse the
endpoint and select a path. Once a path has been selected, the user will be
redirected back to the DNASC. The path will be displayed in a text box.

\subsubsection{Step 3: Name collection}

Finally, the user will be prompted to provide a name for the backup device. 
As long as the name is unique, the backup device will be registered.
This device will be used to identify the backup device in the backups pop-up window.
