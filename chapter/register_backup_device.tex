% Chapter: Register Backup Device Pop-up

\section{Overview}

This chapter will describe the contents of the register backup device pop-up 
window. This pop-up should collect information from the user about a new backup device 
and register it. It is opened by clicking the register backup device button in 
the create backup section of the backups pop-up window. Once registered, the 
backup device can be selected by the user as the destination for a backup transfer.

\subsection{Redirection to Globus web app}

The register backup device pop-up window should redirect the user to the 
\linkglobusbrowseendpoint{browse endpoint ``helper page''} provided by Globus for 
selecting transfer endpoints. This page enables the user to select a Globus collection
and a path on the collection representing a device they wish to register as a backup device. 
Once the user clicks ``Submit'' in the helper page, the user should be redirected back to 
the register backup device pop-up window. A response from the helper page will contain 
data about the backup device selected by the user.

\section{Confirm redirect pop-up\label{sec:confirm-redirect-popup}}

A small pop-up box should appear before the redirect to inform the user of the redirect 
and allow the user to cancel or proceed. If the user chooses to proceed, the box will change 
its contents to the following message to brief the user on the purpose of the redirect:

\vspace{3mm}
``Follow redirection to the Globus web app to select a new backup device. Select a single
folder representing a device, provide a name for the device then click `Submit' to return 
to this page.''
\vspace{3mm}

\section{Register backup device window states}

The register backup device pop-up window should have three states: the initial state,
the incomplete selection state, and the confirm registration state. 

\subsubsection{State transitions}

Transitions between states occur conditionally after the user returns from being redirected 
to the Globus web app. The transitions are as follows:

\begin{itemize}
    \item The initial state is the initial state of the register backup device pop-up window. It
    is not re-entered once it is left.
    \item The incomplete selection state is (re)entered when the user returns from the Globus web app
    without having selected all necessary data for registering a backup device.
    \item The confirm registration state is (re)entered when the user returns from the Globus web app
    having selected all necessary data for registering a backup device.
\end{itemize}

\subsection{Initial state}

The initial state of the register backup device pop-up window should consist of two items:
\begin{itemize}
    \item The string used to brief the user which is included in the description of the confirm 
    redirect pop-up (section \ref{sec:confirm-redirect-popup}).
    \item A button labeled ``Redirect to Globus Web App'' to redirect the user to the helper page.
\end{itemize}

\subsection{Incomplete selection state}

The incomplete selection state of the register backup device pop-up window should inform 
the user that they have not selected all necessary data for registering a backup device.

% start here

\subsection{Select device section}

\noindent To select a new backup device, the user should be redirected to the
Globus web app where they can select a collection and a folder representing a device. 
The select device path subsection should contain a button labeled ``Redirect to Globus Web App'' 
to redirect the user. Before the actual redirect, a pop-up box 
should inform the user of the redirect and allow the user to cancel or proceed. 
\subsection{Confirm registration section}

Once the user has returned from the Globus web app, the confirm registration section should 
display any and all of the following data in a read-only format:

\begin{itemize}\itemsep1pt
    \item ``Globus Collection Name: \texttt{globus\_collection\_name}''
    \item ``Device Path: \texttt{device\_path}''
    \item ``Backup device name: \texttt{backup\_device\_name}''
\end{itemize}

\noindent If any of the above data is missing, the confirm registration section should
include a string to notify the user which data is missing. For example, if the backup device
name and path are missing, the confirm registration section should contain the following string:

\vspace{3mm}
``The following information was not collected: backup device name, device path. Please try 
again by clicking the `Redirect to Globus Web App' button''
\vspace{3mm}

\noindent At the very bottom, the confirm registration section should also contain a button 
labeled ``Register''. The register button should be disabled until a Globus collection, path, 
and backup device name have been selected.

