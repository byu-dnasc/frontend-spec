% Chapter: Register Backup Device Pop-up

\section{Overview}

This chapter will describe the contents of the register backup device pop-up 
window. This pop-up will facilitate the registration of a new backup device
for use in the DNASC. It is opened by clicking the ``Register Backup Device''
button in the ``Create Backup'' subsection of the backups pop-up window.
Once registered, the backup device will be available to be selected by the
user as the destination for a backup transfer.

\subsection{Movitation}
The registration of backup devices simplifies the transfer process in that the 
backup device object serves as an abstraction of the Globus endpoint and path 
which are the transfer destination (or, in the seldom case in which data on 
a backup device needs to be uploaded, the source).

\section{Contents}

The register backup device pop-up window is used to collect information from
the user to identify, validate and register a backup device. The following details 
must be obtained to register a backup device:
\begin{itemize}\itemsep1pt
    \item Name
    \item Globus endpoint id
    \item Globus endpoint path. \emph{Given the endpoint id, there should be some pre-built logic to send the user to
app.globus.org to browse the endpoint and select a path (https://docs.globus.org/modern-research-data-portal/mrdp-description/).}
\end{itemize}

\subsubsection{Step 1: identify backup device}

The first step in registering a backup device is to identify the device. This is 
done by

\section{Implementation}

\noindent Once the path of the device on the endpoint is determined, a ``cookie'' 
should be created and placed on the backup device via Globus Transfer. This cookie 
should consist of an empty file with a unique name. In the event of a download, this 
cookie will be used to identify the backup device.

Upon successful registration, the user-provided name for the backup device and the 
filename of the cookie should be stored in the database. The device names will be 
used to populate a dropdown menu in the ``Create Backup'' pop-up window for the
user to select the backup device to use for the transfer.

Backup device data should be stored in a database on the website backend, not in
the Globus Search index.

