% Chapter: Introduction

\section{Overview}

This is a specification for the frontend component of a new application which will
support the operations of the BYU DNA Sequencing Center (DNASC).

\subsection{DNASC website expansion}

The BYU DNA Sequencing Center provides custom DNA sequencing services to internal and 
external users. Two essential tasks in supporting the provision of these services are 
the sharing and backing up of files generated in the center. The existing internal DNASC 
website \linklsdnasc{lsdnasc.byu.edu} will get a new page to facilitate these tasks.

\subsubsection{Functionalities}

The new webpage will allow users to browse, share and create backups for files located in a 
storage system administered by BYU Office of Research Computing (ORC) using RESTful web 
APIs from Globus, a not-for-profit Platform as a Service (PaaS) provider.

\subsubsection{Architecture}

The new frontend functionalities will interact with these resources:
\begin{itemize}
    \item The DNASC database, the database used by the current DNASC website.
    \item A Globus \linkglobusindex{search index}, a database hosted by Globus and populated 
    by the frontend and backend components of the new application.
    \item Other Globus services accessible by the Globus Search and Globus Transfer APIs.
\end{itemize}

\subsubsection{Appearance}
In general, the new page should be similar in appearance to existing DNASC pages such as 
the ``Plate History'' page. This includes functions such as table filters and a search bar. 

\section{Document organization and content}

The chapters of this document will describe of the appearance and purpose of each of the
elements associated with the new web page. Instructions for implementing the 
functionalities required by each element are provided as appendices to this document. Links
are provided within chapters to appendices which are partilcularly relevant to the element 
being described.

\section{Data models}

A number of data models have been defined to facilitate the implementation of the new
functionality. These models are represented by JSON files and discussed in appendix 
\ref{appendix:models}. Each JSON file consists of a set of properties, each of which 
having a set of key value pairs which describe the property.

\begin{verbatim}
    
{ 
    "properties": {
        "movie_id": { 
            "description": "The movie ID",
            "name": "id",
            "type": "string",
            "pop-up window": "movie movie_id details",
            "in_table": true
        },
        (...)
    }
}

\end{verbatim}